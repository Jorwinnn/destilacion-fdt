\section{Justificación}
La recuperación de gases ligeros en la refinación no sólo incrementa el rendimiento económico de la planta, sino que también responde a criterios de sostenibilidad al aprovechar al máximo recursos de alto valor energético. No obstante, las complejidades del equilibrio de fases y los múltiples fenómenos de transporte hacen que su análisis detallado sea un reto tanto en el ámbito académico como industrial.

Este proyecto provee una solución didáctica y técnica mediante el diseño de una columna industrial respaldada por un modelo matemático y su validación computacional. Con ello se facilita la comprensión de los mecanismos de separación y se permite evaluar de manera cuantitativa el desempeño del proceso antes de una implementación a escala real.

De esta forma, el trabajo aporta un enfoque multidisciplinario que fortalece competencias en modelado, simulación y análisis de operaciones unitarias, preparando al estudiante y al profesional para abordar escenarios reales en la industria de refinación de petróleo.
