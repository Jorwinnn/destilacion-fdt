\section{Conclusión}

Se desarrolló un diseño completo de una columna de destilación multicomponente, específicamente orientado a maximizar la recuperación de gases ligeros (etano, propano, i-butano y n-butano) presentes en el destilado de petróleo crudo. El proyecto integró fundamentos teóricos y herramientas prácticas, permitiendo abordar el sistema desde una visión profunda y aplicada de la ingeniería química.

Se aplicaron de forma articulada los tres fenómenos de transporte (masa, energía y momento), demostrando su interdependencia en la operación estable y eficiente de la columna. A través del desarrollo de un modelo matemático robusto y su resolución numérica, se logró predecir con precisión el comportamiento del sistema bajo condiciones industriales realistas.

Los balances de masa y energía, las relaciones de equilibrio líquido-vapor y las restricciones hidráulicas permitieron establecer los parámetros operativos óptimos, obteniendo resultados que concuerdan con referencias bibliográficas y simulaciones computacionales.

El enfoque numérico empleado, basado en métodos iterativos y criterios estrictos de convergencia, permitió capturar la complejidad del sistema sin recurrir a simplificaciones excesivas, facilitando la adaptación del diseño a distintas condiciones de operación y mezclas de alimentación.

Finalmente, se incluyeron correlaciones para estimar el costo de la columna y su instalación, así como un análisis específico del impacto ambiental asociado al proceso.