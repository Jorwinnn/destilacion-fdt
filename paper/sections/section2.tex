\section{Introducción}
La refinación del petróleo constituye el conjunto de operaciones unitarias destinadas a convertir el crudo en productos de mayor valor comercial. Dentro de dichas operaciones, la destilación multicomponente juega un rol clave al separar fracciones según sus volatibilidades, permitiendo la recuperación de gases ligeros como etano, propano y butano del destilado de petróleo crudo.

El objetivo de este proyecto es diseñar una columna de destilación a escala industrial focalizada en maximizar la recuperación de estos gases ligeros. Para ello se desarrollará un modelo matemático fundamentado en los principios de equilibrio de fases y transporte físico-químico, el cual se implementará en simulaciones computacionales para analizar el comportamiento del sistema y cuantificar su eficiencia separativa.

Este estudio refuerza la vinculación entre teoría y práctica al servir como herramienta formativa en ingeniería química. Además, integra conocimientos de operaciones unitarias y modelado de procesos, ofreciendo una visión aplicada de los desafíos presentes en la industria de refinación.
