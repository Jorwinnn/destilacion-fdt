\section{Marco Teórico}
\subsection{Introducción a los fenómenos de transporte}
Los fenómenos de transporte en sistemas químicos involucran el movimiento de masa, energía y momento entre diferentes regiones del proceso. En una columna de destilación multicomponente estos fenómenos ocurren a escala macroscópica mediante flujos de fluido entre las etapas. Cada etapa de la columna se puede considerar como un volumen de control donde interactúan corrientes de alimentación, reflujo, vapor ascendente y líquido descendente, las cuales transportan masa, entalpía y momento. Estos flujos determinan cómo se distribuyen las concentraciones y las temperaturas a lo largo de la columna.

A escala microscópica, el transporte también se explica mediante mecanismos difusivos: la ley de Fick describe la difusión molecular de materia, la ley de Fourier el flujo de calor por conducción y la ley de Newton las fuerzas viscosas que producen transferencia de momento. Sin embargo, en el enfoque macroscópico propio de los balances globales de la columna, se simplifican estos detalles. En lugar de resolver gradientes locales, se considera que la transferencia neta de masa y energía ocurre por convección entre etapas contiguas. De este modo, las ecuaciones de balance se formulan en términos de flujos integrales: entradas menos salidas igual al cambio o acumulación en cada etapa o en la columna completa, sin necesidad de ecuaciones diferenciales complejas.

\subsubsection{Fundamentos del transporte de masa}
El transporte de masa en la columna de destilación se describe mediante los flujos convectivos entre etapas. Cada etapa recibe corrientes de vapor y de líquido con ciertas composiciones y cede otras con composiciones distintas, de modo que un componente predomina en la fase vapor (ascendente) y otro en la fase líquida (descendente). La separación multicomponente se logra porque cada componente tiende a migrar según su volatilidad, concentrándose preferentemente en la fase apropiada. En el balance global de materia para cada componente en una etapa, las entradas (flujos molares de vapor y líquido entrantes) menos las salidas (flujos salientes) equivalen al cambio de cantidad de esa sustancia en la etapa (acumulación). En régimen estacionario este cambio es cero, por lo que las entradas y salidas se equilibran exactamente.

Este tratamiento macroscópico asume que cada etapa está en equilibrio termodinámico y uniformemente mezclada, de modo que cualquier transferencia de masa entre fase vapor y líquida se refleja en variaciones de los flujos molares. Aunque en detalle la ley de Fick explicaría la difusión molecular de cada componente, en los balances prácticos cada componente se transporta netamente por convección entre etapas. El modelo integra implícitamente el efecto de la difusión microscópica en los valores de los flujos de vapor y líquido, evitándose el uso de gradientes de concentración locales.

\subsubsection{Fundamentos del transporte de energía}
En la columna de destilación, el transporte de energía se manifiesta principalmente como transporte convectivo de entalpía asociado a las corrientes de vapor y líquido. Cada flujo transporta energía térmica: el vapor ascendente lleva entalpía debido a su temperatura (calor sensible) y al calor latente de la vaporización, mientras que el líquido descendente aporta entalpía de su calor sensible (y parte del calor latente al condensarse). En el balance energético de una etapa se contabilizan las entalpías de las corrientes entrantes y salientes; las entradas de entalpía menos las salidas equivalen al cambio de energía almacenada en la etapa. En este balance se incluyen, además, los aportes o extracciones de calor de los equipos auxiliares (reboiler y condensador) como entradas o salidas puntuales.

Este enfoque difiere del análisis de conducción de calor descrito por la ley de Fourier, que opera a nivel microscópico por gradientes de temperatura. En el modelo macroscópico de la columna se considera principalmente la entalpía transportada por los flujos convectivos, reservándose las leyes de conducción para comparaciones conceptuales. El calor suministrado por el reboiler y rechazado por el condensador se incorpora en los balances como entradas o salidas de energía. En la práctica, cuando el vapor condensa sobre el líquido descendente transfiere su calor latente al líquido, y cuando el líquido vaporiza en el reboiler absorbe entalpía, cerrando el ciclo energético de cada etapa. En resumen, la distribución de energía en la columna se caracteriza por los flujos convectivos de entalpía de todas las corrientes, formulando los balances energéticos como entradas menos salidas igual a cambio, sin necesidad de resolver las ecuaciones de conducción térmica.

\subsubsection{Fundamentos del transporte de momento}
El transporte de momento en la columna de destilación está relacionado con las fuerzas ejercidas por el movimiento de los fluidos y sus interacciones en cada etapa. Cada corriente de vapor o de líquido que atraviesa una bandeja posee momento —producto de su masa por su velocidad— y experimenta fuerzas de fricción e inercia. Conceptualmente, el balance de momento se plantea considerando que la diferencia entre el momento entrante y el saliente de una etapa se compensa con las fuerzas externas (por ejemplo, el peso del líquido y la gravedad). Este balance se refleja en la caída de presión que ocurre a través de cada bandeja y a lo largo de la columna, resultado de la fricción del vapor con el líquido y con las superficies internas.

La velocidad y densidad de las corrientes influyen directamente en la magnitud del momento transportado. Un aumento en la velocidad del vapor incrementa el impulso que ejerce sobre el líquido, lo cual produce una mayor caída de presión. Este efecto explica los límites hidráulicos en la operación de la columna: un flujo de vapor excesivo genera un arrastre de líquido hacia arriba, provocando la \emph{inundación} (flooding), donde la bandeja se llena de líquido y la presión ascendente se incrementa drásticamente. Por el contrario, un flujo de vapor insuficiente no puede sostener el líquido en las bandejas, provocando \emph{weeping} (goteo) y disminuyendo la eficiencia de separación. En ambos casos las alteraciones en el transporte de momento repercuten en los balances de masa y energía, pues modifican los flujos efectivos que atraviesan las etapas.

\subsection{Interacción entre los fenómenos de transporte en procesos industriales}
En la operación de una columna de destilación multicomponente, los fenómenos de transporte de masa, energía y momento están íntimamente acoplados. Los flujos moleculares de materia transportan simultáneamente energía en forma de entalpía y momento en forma de impulso de fluido. Por ejemplo, aumentar el flujo de reflujo líquido incrementa la cantidad de materia de componentes ligeros que asciende hacia etapas superiores, pero al mismo tiempo introduce mayor entalpía de líquido frío y modifica la caída de presión en la columna. De modo similar, elevar la potencia del reboiler aumenta el vapor ascendente (más transporte de masa) y la entalpía que circula, pero también incrementa la presión y la energía total en movimiento.

El diseño y la operación de la columna deben equilibrar estos tres balances. Los balances globales de masa y energía determinan la distribución de composición y temperatura en cada etapa, mientras que el balance de momento impone condiciones hidráulicas de operación. En la práctica, cualquier variación en los flujos afecta simultáneamente los tres balances: un cambio en la relación de reflujo o en la carga de alimentación alterará las corrientes de materia y energía y, a la vez, modificará la caída de presión. Fenómenos límite como la inundación y el goteo ejemplifican esta interacción: resultan de combinaciones extremas de flujos de vapor y líquido que rompen las condiciones óptimas de operación.

En resumen, los tres fenómenos de transporte deben analizarse de manera conjunta en el diseño de la columna. Los balances se formulan a nivel global por etapa o para la columna entera según el principio entradas menos salidas igual a cambio, involucrando variables como flujos molares, entalpías y velocidades del fluido. Esta perspectiva integral permite predecir el comportamiento de la columna y asegurar condiciones óptimas de recuperación de gases ligeros en el destilado de petróleo crudo.
