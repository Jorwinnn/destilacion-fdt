\subsection{Impacto ambiental}

El \textbf{Potencial de Calentamiento Global (GWP)} es un índice que mide la capacidad de un gas para atrapar calor en la atmósfera en comparación con el CO$_2$ durante un período de referencia (generalmente 100 años). Cuanto mayor es el GWP, más contribuye ese gas al calentamiento global por unidad de masa emitida.

Los compuestos con los que opera la columna de destilación (etano, propano, n-butano e i-butano) presentan un GWP individual que varía aproximadamente entre $2{,}8$ y $5{,}5$ (considerando un horizonte de 100 años). Por lo tanto, si se producen fugas y estos hidrocarburos se liberan sin combustión, cada kilogramo emitido contribuye con entre $2{,}8$ y $5{,}5$ kg de CO$_2$ equivalente al balance de gases de efecto invernadero, intensificando el cambio climático y sus consecuencias (aumento de temperaturas, alteración de patrones climáticos, derretimiento de glaciares, etc.).

\subsubsection*{Contribución conjunta}

\begin{itemize}
    \item \textbf{Huella de carbono industrial.} La suma de fugas fugitivas de etano, propano, n-butano e i-butano eleva la huella de carbono equivalente del proceso, ya que cada fuga aporta CO$_2$ equivalente según su GWP y el volumen manejado.
    \item \textbf{Smog fotoquímico.} Al comportarse todos ellos como compuestos orgánicos volátiles (COV), la mezcla del GLP en la atmósfera favorece las reacciones fotoquímicas que generan ozono troposférico y componentes de smog, deteriorando la calidad del aire y afectando salud respiratoria y vegetación.
    \item \textbf{Riesgo de ignición.} La estrecha ventana inflamable de la mezcla y su tendencia a acumularse en zonas bajas o recintos cerrados incrementan la posibilidad de ignición y explosiones, con liberación masiva de hidrocarburos al ambiente.
\end{itemize}