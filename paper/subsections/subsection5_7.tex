\subsection{Resultados de la resolución del modelo matemático}
Para el diseño de la columna de destilación se partió de una alimentación líquida saturada de 200 kmol/h con fracciones molares, en fase líquida, de etano (4.1 \%), propano (62 \%), i-butano (16.6 \%) y n-butano (17.3 \%). Las condiciones de operación consideraron una presión uniforme de 101 325 Pa tanto en el condensador como en el rehervidor, y una temperatura inicial de cálculo del punto de burbuja de 250 K. La relación de reflujo real se fijó en 1.5 veces el mínimo, $R/R_{\mathrm{min}} = 1.5$, aplicando una eficiencia de platos $\eta = 0.5$ según la expresión:
$$
    R_{\mathrm{real}} = \frac{R}{\eta}.
$$

Mediante la aplicación de las correlaciones de Fenske, Underwood, Gilliland y Kirkbride, se determinaron inicialmente 21 platos teóricos, con el punto de alimentación en la posición 12. El balance de caudales arrojó un flujo de destilado $D = 166\ \text{kmol/h}$ y un flujo de fondos $B = 34\ \text{kmol/h}$. Asimismo, la relación de reflujo mínimo resultó $R_{\mathrm{min}} = 0.436$, la relación teórica $R = 0.653$ y la relación real $R_{\mathrm{real}} = 1.307$.
\bigskip

Al incorporar los efectos del condensador y del rehervidor, el diseño final se ajustó a 23 platos, desplazando la alimentación al plato número 13. En cuanto a la configuración hidráulica y geométrica de los platos, cada uno presenta un diámetro de 1.5 m y un espaciamiento vertical de 0.85 m. El área activa corresponde al 80 \% del área total del plato y, de ésta, el 20 \% se destina a 400 orificios de paso, cada uno con diámetro equivalente al 2 \% del diámetro del plato. La altura del canal de desagüe (downcomer) es de 0.05 m, con una lámina líquida sobre el mismo de 0.05 m.

\bigskip
Nuestra simulación numérica convergió en 85 iteraciones principales.

\newpage
\begin{table}[ht]
    \centering\small
    \caption{Temperaturas, presiones y flujos de líquido y vapor en cada nivel de plato}
    \begin{tabular}{|p{3cm}|p{3cm}|p{3cm}|p{3cm}|}
        \hline
        \textbf{T [K]} & \textbf{P [Pa]} & \textbf{L [kmol/h]} & \textbf{V [kmol/h]} \\ \hline
        230.19         & 101325.00       & 216.96              & 0.00                \\ \hline
        242.45         & 101850.15       & 197.80              & 383.03              \\ \hline
        249.06         & 102388.44       & 188.70              & 363.80              \\ \hline
        252.29         & 102932.39       & 185.38              & 354.70              \\ \hline
        253.77         & 103478.89       & 183.84              & 351.38              \\ \hline
        254.59         & 104026.89       & 182.82              & 349.84              \\ \hline
        255.20         & 104576.02       & 181.99              & 348.82              \\ \hline
        255.72         & 105126.11       & 181.27              & 347.99              \\ \hline
        256.19         & 105677.06       & 180.65              & 347.27              \\ \hline
        256.61         & 106228.73       & 180.12              & 346.65              \\ \hline
        256.98         & 106781.00       & 179.69              & 346.12              \\ \hline
        257.31         & 107333.76       & 179.33              & 345.69              \\ \hline
        257.60         & 107886.92       & 179.06              & 345.33              \\ \hline
        257.85         & 108440.38       & 359.51              & 345.06              \\ \hline
        265.47         & 109003.14       & 355.80              & 325.51              \\ \hline
        268.81         & 109569.83       & 354.81              & 321.80              \\ \hline
        270.34         & 110138.46       & 354.10              & 320.81              \\ \hline
        271.23         & 110708.37       & 353.45              & 320.10              \\ \hline
        271.90         & 111279.28       & 352.88              & 319.45              \\ \hline
        272.46         & 111851.02       & 352.41              & 318.88              \\ \hline
        272.95         & 112423.45       & 352.03              & 318.41              \\ \hline
        273.38         & 112996.45       & 351.75              & 318.03              \\ \hline
        273.75         & 113569.93       & 34.00               & 317.75              \\ \hline
    \end{tabular}
\end{table}

\newpage
\begin{table}[ht]
    \centering\small
    \caption{Densidades de fase y velocidades de vapor en cada nivel de plato}
    \begin{tabular}{|p{2.5cm}|p{2.5cm}|p{2.5cm}|p{2.5cm}|p{2.5cm}|}
        \hline
        \textbf{$\rho_L$ [kg/m$^3$]} & \textbf{$\rho_V$ [kg/m$^3$]} & \textbf{$u_{\text{min}}$ [m/s]} & \textbf{$u$ [m/s]} & \textbf{$u_{\text{max}}$ [m/s]} \\ \hline
        511.12                       & 2.15                         & 0.00                            & 0.00               & 0.00                            \\ \hline
        533.23                       & 2.34                         & 0.97                            & 1.49               & 1.61                            \\ \hline
        546.67                       & 2.42                         & 0.95                            & 1.45               & 1.60                            \\ \hline
        552.46                       & 2.47                         & 0.94                            & 1.42               & 1.60                            \\ \hline
        555.06                       & 2.49                         & 0.93                            & 1.41               & 1.59                            \\ \hline
        556.59                       & 2.51                         & 0.93                            & 1.40               & 1.59                            \\ \hline
        557.75                       & 2.52                         & 0.93                            & 1.39               & 1.59                            \\ \hline
        558.73                       & 2.53                         & 0.93                            & 1.38               & 1.59                            \\ \hline
        559.59                       & 2.54                         & 0.93                            & 1.38               & 1.59                            \\ \hline
        560.33                       & 2.55                         & 0.93                            & 1.37               & 1.58                            \\ \hline
        560.95                       & 2.56                         & 0.93                            & 1.36               & 1.58                            \\ \hline
        561.45                       & 2.57                         & 0.93                            & 1.35               & 1.58                            \\ \hline
        561.85                       & 2.58                         & 0.93                            & 1.35               & 1.58                            \\ \hline
        562.16                       & 2.59                         & 0.93                            & 1.34               & 1.57                            \\ \hline
        571.72                       & 2.74                         & 0.91                            & 1.30               & 1.54                            \\ \hline
        575.75                       & 2.81                         & 0.90                            & 1.29               & 1.53                            \\ \hline
        577.74                       & 2.84                         & 0.90                            & 1.29               & 1.52                            \\ \hline
        579.04                       & 2.85                         & 0.90                            & 1.28               & 1.52                            \\ \hline
        580.06                       & 2.86                         & 0.90                            & 1.28               & 1.52                            \\ \hline
        580.90                       & 2.87                         & 0.90                            & 1.27               & 1.52                            \\ \hline
        581.61                       & 2.88                         & 0.90                            & 1.26               & 1.52                            \\ \hline
        582.20                       & 2.89                         & 0.90                            & 1.26               & 1.52                            \\ \hline
        582.68                       & 2.90                         & 0.90                            & 1.25               & 1.51                            \\ \hline
    \end{tabular}
\end{table}

\newpage
\begin{table}[ht]
    \centering\small
    \caption{Fracciones molares líquidas de etano, propano, i-butano y n-butano por plato}
    \begin{tabular}{|p{2.5cm}|p{2.5cm}|p{2.5cm}|p{2.5cm}|}
        \hline
        \textbf{$x_{\text{etano}}$} & \textbf{$x_{\text{propano}}$} & \textbf{$x_{\text{i-butano}}$} & \textbf{$x_{\text{n-butano}}$} \\ \hline
        0.05                        & 0.75                          & 0.19                           & 0.02                           \\ \hline
        0.01                        & 0.48                          & 0.45                           & 0.06                           \\ \hline
        0.00                        & 0.30                          & 0.59                           & 0.12                           \\ \hline
        0.00                        & 0.22                          & 0.61                           & 0.17                           \\ \hline
        0.00                        & 0.20                          & 0.59                           & 0.21                           \\ \hline
        0.00                        & 0.19                          & 0.56                           & 0.25                           \\ \hline
        0.00                        & 0.18                          & 0.52                           & 0.30                           \\ \hline
        0.00                        & 0.18                          & 0.48                           & 0.33                           \\ \hline
        0.00                        & 0.18                          & 0.45                           & 0.37                           \\ \hline
        0.00                        & 0.18                          & 0.42                           & 0.40                           \\ \hline
        0.00                        & 0.17                          & 0.40                           & 0.42                           \\ \hline
        0.00                        & 0.17                          & 0.38                           & 0.44                           \\ \hline
        0.00                        & 0.17                          & 0.36                           & 0.46                           \\ \hline
        0.00                        & 0.17                          & 0.35                           & 0.47                           \\ \hline
        0.00                        & 0.06                          & 0.36                           & 0.59                           \\ \hline
        0.00                        & 0.02                          & 0.32                           & 0.66                           \\ \hline
        0.00                        & 0.00                          & 0.27                           & 0.72                           \\ \hline
        0.00                        & 0.00                          & 0.22                           & 0.77                           \\ \hline
        0.00                        & 0.00                          & 0.18                           & 0.82                           \\ \hline
        0.00                        & 0.00                          & 0.14                           & 0.86                           \\ \hline
        0.00                        & 0.00                          & 0.11                           & 0.89                           \\ \hline
        0.00                        & 0.00                          & 0.08                           & 0.92                           \\ \hline
        0.00                        & 0.00                          & 0.06                           & 0.94                           \\ \hline
    \end{tabular}
\end{table}

\newpage
\begin{table}[ht]
    \centering\small
    \caption{Fracciones vaporizadas de etano, propano, i-butano y n-butano por plato}
    \begin{tabular}{|p{2.5cm}|p{2.5cm}|p{2.5cm}|p{2.5cm}|}
        \hline
        \textbf{$y_{\text{etano}}$} & \textbf{$y_{\text{propano}}$} & \textbf{$y_{\text{i-butano}}$} & \textbf{$y_{\text{n-butano}}$} \\ \hline
        0.28                        & 0.67                          & 0.04                           & 0.00                           \\ \hline
        0.05                        & 0.75                          & 0.19                           & 0.02                           \\ \hline
        0.03                        & 0.60                          & 0.33                           & 0.04                           \\ \hline
        0.02                        & 0.51                          & 0.40                           & 0.07                           \\ \hline
        0.02                        & 0.47                          & 0.41                           & 0.09                           \\ \hline
        0.02                        & 0.46                          & 0.40                           & 0.12                           \\ \hline
        0.02                        & 0.45                          & 0.38                           & 0.14                           \\ \hline
        0.02                        & 0.45                          & 0.36                           & 0.16                           \\ \hline
        0.02                        & 0.45                          & 0.34                           & 0.18                           \\ \hline
        0.02                        & 0.45                          & 0.33                           & 0.20                           \\ \hline
        0.02                        & 0.45                          & 0.31                           & 0.22                           \\ \hline
        0.02                        & 0.45                          & 0.30                           & 0.23                           \\ \hline
        0.02                        & 0.45                          & 0.29                           & 0.24                           \\ \hline
        0.02                        & 0.45                          & 0.28                           & 0.25                           \\ \hline
        0.00                        & 0.19                          & 0.38                           & 0.43                           \\ \hline
        0.00                        & 0.06                          & 0.39                           & 0.55                           \\ \hline
        0.00                        & 0.02                          & 0.35                           & 0.63                           \\ \hline
        0.00                        & 0.01                          & 0.29                           & 0.70                           \\ \hline
        0.00                        & 0.00                          & 0.24                           & 0.76                           \\ \hline
        0.00                        & 0.00                          & 0.19                           & 0.81                           \\ \hline
        0.00                        & 0.00                          & 0.15                           & 0.85                           \\ \hline
        0.00                        & 0.00                          & 0.12                           & 0.88                           \\ \hline
        0.00                        & 0.00                          & 0.09                           & 0.91                           \\ \hline
    \end{tabular}
\end{table}

\newpage
\subsubsection{Análisis de resultados}
Los resultados numéricos obtenidos permiten validar el comportamiento interno de la columna de destilación, así como evaluar la coherencia físico-química del diseño propuesto. A continuación, se realiza un análisis detallado de cada conjunto de variables:

\begin{enumerate}
    \item \textbf{Temperatura y presión:}\\
          Las temperaturas en la columna aumentan desde \SI{230.19}{\kelvin} en el plato superior hasta \SI{273.75}{\kelvin} en el plato inferior. Este comportamiento es completamente consistente con el gradiente térmico característico de una columna de destilación: el condensador enfría el tope mientras que el rehervidor calienta el fondo. Además, las presiones aumentan ligeramente de \SI{101325}{\pascal} hasta \SI{113569.93}{\pascal}, lo cual concuerda con las caídas de presión acumuladas por capilaridad y flujos a través de platos y orificios. La magnitud del incremento (aproximadamente \SI{12245}{\pascal}) es razonable para una columna de 23 etapas con espaciado de \SI{0.85}{m} entre platos, y sugiere un diseño hidráulico sin excesivas pérdidas de carga.

    \item \textbf{Flujos de líquido y vapor:}\\
          El flujo descendente de líquido $L$ disminuye de \SI{216.96}{\kilo\mole\per\hour} en la parte superior a \SI{34}{\kilo\mole\per\hour} en el plato 23, mientras que el flujo ascendente de vapor $V$ muestra la tendencia opuesta, aumentando desde 0 hasta \SI{317.75}{\kilo\mole\per\hour}. Esta evolución es coherente con la entrada neta de alimentación en el plato 13 y la contribución del rehervidor, que aporta vapor al sistema, mientras que el condensador extrae vapor en el tope. La relación de estos flujos con los valores de destilado ($D=$ \SI{166}{\kilo\mole\per\hour}) y fondos ($B=$ \SI{34}{\kilo\mole\per\hour}) valida las condiciones de diseño: la suma de $D$ y $B$ coincide con el caudal alimentado de \SI{200}{\kilo\mole\per\hour}, y el punto de inflexión de los flujos (plato 13) corresponde exactamente con la ubicación de la alimentación, lo que demuestra consistencia en el balance de materia.

          \newpage
    \item \textbf{Densidades de fase:}\\
          Las densidades del líquido $\rho_L$ aumentan progresivamente desde \SI{511.12}{\kilogram\per\cubic\meter} en el plato superior hasta \SI{582.68}{\kilogram\per\cubic\meter} en el plato inferior, lo que refleja la acumulación de componentes más pesados (n-butano e i-butano) en la base, como es esperado en procesos de separación por volatilidad. Por otro lado, la densidad de vapor $\rho_V$ también aumenta ligeramente (de \SI{2.15}{\kilogram\per\cubic\meter} a \SI{2.90}{\kilogram\per\cubic\meter}), sugiriendo un enriquecimiento del vapor en especies más pesadas conforme desciende por la columna. Esta variación es físicamente plausible, dado que el etano (el componente más volátil) abandona mayormente por el tope.

    \item \textbf{Velocidades de vapor:}\\
          Las velocidades reales $u$ de vapor se mantienen entre \SI{1.25}{m/s} y \SI{1.49}{m/s}, siempre dentro del intervalo permisible entre $u_{\text{min}}$ y $u_{\text{max}}$ calculado por criterios de diseño hidráulico. Esta observación indica que no se presentan fenómenos no deseados como goteo (weeping) ni inundación (flooding). Además, la tendencia decreciente de $u$ se relaciona directamente con al decremento del flujo de vapor.

    \item \textbf{Fracciones molares líquidas y vaporizadas:}\\
          En las etapas altas (platos 1-4) la fase líquida está enriquecida en etano y propano, con $x_{\text{etano}}$ decreciendo desde 0.05 hasta 0.00 y $x_{\text{propano}}$ desde 0.75 hasta 0.22, mientras que $x_{\text{n-butano}}$ aumenta de 0.02 a 0.17. Esto refleja la volatilidad relativa de estos componentes y confirma que el etano se arrastra preferentemente hacia la cabeza de la columna. Por su parte, las fracciones vaporizadas $y_{\text{etano}}$ y $y_{\text{propano}}$ en esas mismas etapas disminuyen de 0.28 a 0.02 y de 0.67 a 0.51, respectivamente, indicando un equilibrio sólido entre fases.

          Hacia el centro de la columna (platos 5-15), $x_{\text{i-butano}}$ alcanza su máximo (0.61 en el plato 4-5) y luego decrece gradualmente hasta 0.35, mientras que $x_{\text{n-butano}}$ sube de 0.21 a 0.47, lo que es característico de la zona de stripping intermedia. Simultáneamente, las $y$ correspondientes muestran comportamientos análogos: $y_{\text{i-butano}}$ cae de 0.40 a 0.28 y $y_{\text{n-butano}}$ aumenta de 0.07 a 0.25, confirmando que el vapor extrae progresivamente los butanos hacia abajo.

          En las etapas bajas (platos 16-23), casi todo el etano y propano han sido removidos ($x_{\text{etano}}=x_{\text{propano}}=0$), mientras que $x_{\text{i-butano}}$ y $x_{\text{n-butano}}$ se estabilizan alrededor de 0.06-0.27 y 0.72-0.94, respectivamente. Las fracciones vaporizadas finales indican un $y_{\text{n-butano}}$ superior al 0.91, demostrando que el fondo se enriquece casi exclusivamente en n-butano. Este perfil concuerda con las expectativas teóricas de fraccionamiento multicomponente y valida la posición de alimentación y la relación de reflujo elegidas.
\end{enumerate}

En conjunto, los datos presentados reflejan un diseño coherente desde el punto de vista termodinámico, hidrodinámico y estructural. La progresión esperada de temperaturas, presiones, flujos, densidades y velocidades respeta los principios fundamentales de la destilación fraccionada, y las cifras resultantes se encuentran dentro de rangos técnicos y operativos típicos para sistemas multicomponente bajo condiciones industriales moderadas. Por lo tanto, se puede concluir que el modelo y sus parámetros conducen a un comportamiento físico realista y técnicamente viable.
