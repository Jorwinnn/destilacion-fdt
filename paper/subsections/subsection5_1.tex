\subsection{Objeto de estudio}
En el corazón del procesamiento contemporáneo de petróleo crudo yace la imperiosa necesidad de una recuperación eficiente de sus fracciones más volátiles y económicamente valiosas: los gases ligeros. Específicamente, la separación selectiva de etano, propano, i-butano y n-butano no solo representa una optimización crucial del rendimiento económico, sino que también se alinea con imperativos de sostenibilidad al maximizar el aprovechamiento de recursos energéticos de alta densidad. Sin embargo, la inherente complejidad del equilibrio de fases y la intrincada interacción de los fenómenos de transporte convierten esta tarea en un desafío técnico y académico considerable.

El presente proyecto acomete este desafío mediante el \textbf{diseño riguroso de una columna de destilación multicomponente a escala industrial}, concebida explícitamente para la recuperación optimizada de dichos gases ligeros a partir del destilado primario de petróleo. Este estudio se distingue por su profundo énfasis en la \textbf{comprensión y modelado de los fenómenos de transporte subyacentes} —transferencia de masa, energía y momento— que dictan la viabilidad y eficiencia del proceso separativo.

Así, el objeto de estudio trasciende el mero dimensionamiento del equipo; se adentra en el \textbf{análisis detallado del comportamiento fisicoquímico e hidrodinámico} dentro de la columna. Se busca desentrañar y cuantificar los mecanismos fundamentales que gobiernan la separación de estas mezclas multicomponentes complejas, absteniéndose deliberadamente de los procesos subsecuentes de tratamiento de las corrientes para concentrar el análisis en la unidad de destilación. De esta forma, se aspira a generar un conocimiento aplicable que no solo resuelva un problema industrial concreto, sino que también sirva como una herramienta formativa robusta en la ingeniería química.
